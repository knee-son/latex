\documentclass{article}
\usepackage{amsmath, amssymb}
\usepackage[margin=1in]{geometry}

\begin{document}

\section*{Common Shorthand Notations in Mathematics}

\begin{itemize}
    \item \textbf{iff}: if and only if
    \item \textbf{s.t.}: such that
    \item \textbf{w.r.t.}: with respect to
    \item \textbf{i.e.}: that is (from Latin \textit{id est})
    \item \textbf{e.g.}: for example (from Latin \textit{exempli gratia})
    \item \textbf{w.l.o.g.}: without loss of generality
    \item \textbf{a.e.}: almost everywhere
    \item \textbf{a.s.}: almost surely
    \item \textbf{s.a.}: see also
    \item \textbf{Q.E.D.}: which was to be demonstrated (from Latin \textit{quod erat demonstrandum}), typically used at the end of a proof
    \item \textbf{s.i.}: similarly (sometimes used to avoid repetition in proofs)
    \item \textbf{RHS}: right-hand side
    \item \textbf{LHS}: left-hand side
    \item \textbf{w.p.}: with probability
    \item \textbf{w.h.p.}: with high probability
    \item \textbf{$\forall$}: for all (universal quantifier)
    \item \textbf{$\exists$}: there exists (existential quantifier)
    \item \textbf{$\subseteq$}: is a subset of
    \item \textbf{$\supseteq$}: is a superset of
    \item \textbf{$\subset$}: is a proper subset of
    \item \textbf{$\supset$}: is a proper superset of
    \item \textbf{$\cup$}: union
    \item \textbf{$\cap$}: intersection
    \item \textbf{$\infty$}: infinity
    \item \textbf{$\mathbb{N}$}: the set of natural numbers
    \item \textbf{$\mathbb{Z}$}: the set of integers
    \item \textbf{$\mathbb{Q}$}: the set of rational numbers
    \item \textbf{$\mathbb{R}$}: the set of real numbers
    \item \textbf{$\mathbb{C}$}: the set of complex numbers
\end{itemize}

\end{document}
